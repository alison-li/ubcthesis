%% The following is a directive for TeXShop to indicate the main file
%%!TEX root = diss.tex

\chapter{Abstract}

% The abstract is a concise and accurate summary of the scholarly work described in the document. 
% It states the problem, the methods of investigation, and the general conclusions, and should not contain tables, graphs, 
% complex equations, or illustrations. 
% There is a single scholarly abstract for the entire work, and it must not exceed 350 words in length.

Software developers traverse several commits and issues from issue tracking systems 
when exploring software revision history in order to unearth questions about the rationale behind presently written code.
This exploration results in a cognitive and temporal burden as developers
context switch between their \entity{IDE}s and a project's issue tracking system to locate code rationale information
associated with each commit.

To reduce the costs of software revision history exploration,
we introduce an approach using regular expression patterns to reduce the search space of commits in a revision history to signal to a developer which commits in a history merit further investigation.
Additionally, our approach minimizes the distance between a developer's \entity{IDE} and issues from an \entity{ITS}
by directly integrating issue information in the same context.
We implemented these techniques in Intelligent History, a plugin for the IntelliJ IDEA \entity{IDE}.

To evaluate the Intelligent History, we conducted a controlled laboratory study and recruited 10 software developers.
We asked two sets of questions related to the intent behind source code, requiring the participants to explore 
commit histories for two Java classes from the open-source Apache Kafka project.
We also conduct a semi-structured interview on the participants' experiences with examining issues and commits 
and their use of Intelligent History in the experiment.

The results from our analysis shows:
(1) there was overlap between at least half of the commits
that participants examined without Intelligent History and the commits that Intelligent History
highlighted; (2) participants who used Intelligent History examined significantly
less commits than participants who did not use Intelligent History;
and (3) direct integration of issue information in the \entity{IDE}
with Intelligent History might support developers who employ a commit history exploration approach
we describe as \textit{linear} or \textit{cyclic and backtracking} to contextualize
source code changes quickly and view slightly more issues than if they did not have
direct issue integration.
Our analysis demonstrates promise for the design and approach in Intelligent History
for supporting developers in more effectively exploring a commit history to answer code rationale questions.

% Consider placing version information if you circulate multiple drafts
% \vfill
% \begin{center}
% \begin{sf}
% \fbox{Revision: \today}
% \end{sf}
% \end{center}