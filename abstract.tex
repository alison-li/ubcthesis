%% The following is a directive for TeXShop to indicate the main file
%%!TEX root = diss.tex

\chapter{Abstract}

% The abstract is a concise and accurate summary of the scholarly work described in the document. 
% It states the problem, the methods of investigation, and the general conclusions, and should not contain tables, graphs, 
% complex equations, or illustrations. 
% There is a single scholarly abstract for the entire work, and it must not exceed 350 words in length.

Software developers traverse several commits and issues from issue tracking systems (\entity{ITS})
when exploring software revision history in order to unearth questions about intent and rationale behind presently written code.
This exploration of revision history and issues results in a cognitive and temporal burden as developers
switch back-and-forth between their \entity{IDE}s and a project's \entity{ITS} to locate code rationale information
associated with each commit.
We hypothesize identifying important commits automatically and providing integrated access between commit and issue information could help a developer more effectively access the rationale behind code.
To reduce the cognitive and temporal costs of software revision history exploration,
we introduce an approach using regular expression patterns to reducing the search space of commits in a revision history to help
signal to a developer beforehand which commits in a history may be worth further investigating.
Additionally, our approach minimizes the distance between a developer's \entity{IDE} and relevant issues from an \entity{ITS}
by presenting a view of the issue information in the same context.
We implemented these techniques in Intelligent History, a plugin for the IntelliJ IDEA \entity{IDE}.
To evaluate the plugin, we conducted a laboratory user study in the form of a semi-structured interview and 
recruited 10 participants, of which 6 were professional developers and 4 were students.
We asked two sets of questions related to the intent behind code and requring the exploration of 
commit histories found in the open-source Apache Kafka project and a series of follow-up interview questions about the participants'
experiences with examining issues and commits and their use of the plugin in the experiment.
The results from our analysis shows that: \dots

% Consider placing version information if you circulate multiple drafts
\vfill
\begin{center}
\begin{sf}
\fbox{Revision: \today}
\end{sf}
\end{center}
