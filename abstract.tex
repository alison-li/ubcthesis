%% The following is a directive for TeXShop to indicate the main file
%%!TEX root = diss.tex

\chapter{Abstract}

% The abstract is a concise and accurate summary of the scholarly work described in the document. 
% It states the problem, the methods of investigation, and the general conclusions, and should not contain tables, graphs, 
% complex equations, or illustrations. 
% There is a single scholarly abstract for the entire work, and it must not exceed 350 words in length.

Software developers traverse several commits and issues from issue tracking systems 
when exploring software revision history \rev{to answer questions} about the rationale behind presently written code.
\rev{Existing tools impose a cognitive burden on developers as developers 
must sift through many commits and must transition between commit information 
and issue tracking information presented separately.}
Developers also must context switch between their \entity{IDE}s and a project's 
issue tracking system to locate code rationale information associated with each commit.
\rev{More effective support for software revision history exploration would 
reduce the cognitive burden for developers, allowing them to answer code 
rationale questions in the limited time available for tasks.}

We introduce Intelligent History\rev{, which uses} commit history highlighting to reduce the search space of 
commits in a revision history, recommending to a developer which commits in a history might merit further investigation.
Additionally, \rev{Intelligent History} minimizes the distance between a developer's \entity{IDE} and issues from an issue tracking system
by directly integrating issue information in the same context.

To evaluate the Intelligent History, we conducted a controlled laboratory study and recruited 10 software developers.
We asked two sets of questions related to the intent behind source code, requiring the participants to explore 
commit histories for two Java classes from the open-source Apache Kafka project.
We also conduct a semi-structured interview on the participants' experiences with examining issues and commits 
and their use of Intelligent History in the experiment.

The results from our analysis shows:
(1) participants who used Intelligent History examined significantly
less commits than participants who did not use Intelligent History
\rev{while producing correct answers to questions posed};
(2) direct integration of issue information in the \entity{IDE}
with Intelligent History \rev{supports} developers who employ a commit history exploration approach
we describe as \textit{linear} or \textit{cyclic and backtracking} \rev{to} view slightly more issues than if they did not have
direct issue integration; and
(3) \rev{our heuristics were accurate to the extent that there was} overlap between at least half of the commits
that participants examined without Intelligent History and the commits that Intelligent History
highlighted.

% Consider placing version information if you circulate multiple drafts
% \vfill
% \begin{center}
% \begin{sf}
% \fbox{Revision: \today}
% \end{sf}
% \end{center}