%% The following is a directive for TeXShop to indicate the main file
%%!TEX root = diss.tex

%% https://www.grad.ubc.ca/current-students/dissertation-thesis-preparation/preliminary-pages
%% 
%% LAY SUMMARY Effective May 2017, all theses and dissertations must
%% include a lay summary.  The lay or public summary explains the key
%% goals and contributions of the research/scholarly work in terms that
%% can be understood by the general public. It must not exceed 150
%% words in length.

\chapter{Lay Summary}

This thesis investigates how identifying non-important and important 
revisions in a software system's revision history can help 
software developers to better answer questions about the motivation behind code.
Developers also may use issues, which are non-source code artifacts tied to revisions, 
as a source of code rationale information.
We introduce a plugin for IntelliJ IDEA that consists of two approaches to address this.
First, the approach uses the lines of code changes in a revision
to determine if the revision solely contains code changes that the developer
might consider as non-essential.
Second, the approach attempts to reduce context switching between an \entity{IDE}
and an application for viewing issue information, providing the developer with
better access to an issue along with other metrics about the issue.
To evaluate the plugin, we deployed a controlled laboratory study.
We collected data from a total of 10 software developers.