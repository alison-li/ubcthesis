%% This is the original intro.tex file. 
%% The following is a directive for TeXShop to indicate the main file
%%!TEX root = diss.tex

\chapter{Introduction}
\label{ch:Introduction}

%%%%%%%%%%%%%%%%%%%%%%%%%%%%%%%%%%%%%%%%%%%%%%%%%%%%%%%%%%%%%%%%%%%%%%

LaToza et al. \cite{latoza_maintaining_2006} found the most serious problem software developers face is understanding code rationale, followed by Ko et al. \cite{ko_information_2007} reporting the intent behind code being the most difficult information need to satisfy. 
In further agreement, LaToza and Myers \cite{latoza_hard-answer_2010} unearth code rationale as the single most frequently reported “hard-to-answer” question category in software development. 
Code rationale includes questions not clearly answered by examining code itself, such as algorithm choice, optimization choice, and hidden criteria that motivate design choices \cite{latoza_hard-answer_2010}. 
Though developers often choose to forego investigating design documents due to difficulties and the cost of locating them and instead seek out knowledgeable colleagues \cite{latoza_maintaining_2006}, this practice becomes unsustainable in projects and companies at risk of turnover. 
The danger of relying solely on colleagues for code rationale information is illustrated by LaToza et al. 's assumption of the existence of a quintessential ``team historian'' in a software development team \cite{latoza_maintaining_2006}:

\begin{quote}
Almost all teams have a team historian who is the go-to person for questions about the
code. Often this person is the developer lead and has been with the code base the longest.
\end{quote}

Confronted with turnover-induced knowledge loss, Robillard \cite{robillard_turnover-induced_2021} emphasizes developers leveraging artifacts from project history after a colleague’s departure to understand developer intent behind code a departed colleague wrote. 
It therefore becomes all the more important to reduce the cost for developers in using artifacts from project history in order to obtain code rationale. These artifacts can include commit history and information from issue tracking systems.

% Comment on existing tools like git command-line usage and the integration of git usage tools in popular editors like VS Code and IntelliJ

% There is also interesting work that we will discuss in related work section on commit history exploration and issue tracking system information integration

% In this thesis, we investigate ...

% To investigate the usefulness of ...

% This thesis makes X contributions:
\begin{itemize}
	\item TODO
\end{itemize}

% This thesis is structured as follows.

%%%%%%%%%%%%%%%%%%%%%%%%%%%%%%%%%%%%%%%%%%%%%%%%%%%%%%%%%%%%%%%%%%%%%%
\endinput

Any text after an \endinput is ignored.
You could put scraps here or things in progress.
