\chapter{Evaluation}
\label{ch:Evaluation}

We hypothesized identifying important commits automatically and providing integrated access between commit and issue information could help a developer more effectively access the rationale for code.
To evaluate our hypothesis, we conducted a user study in the form of a semi-structured interview and devised the following research questions:

\begin{itemize}[leftmargin=*]
    \item[] \label{itm:RQ1} \textbf{Research Question 1 (\entity{RQ1}):} \textit{Does highlighting important commits help software developers explore a file’s revision history more efficiently?}
    \item[] \label{itm:RQ2} \textbf{Research Question 2 (\entity{RQ2}):} \textit{Does highlighting important commits cause software developers to be more efficient at exploring a file’s revision history than a conventional linear, temporal, flat view of revision history?}
    \item[] \label{itm:RQ3} \textbf{Research Question 3 (\entity{RQ3}):} \textit{How useful is direct integration of issue information in the IDE for helping a developer search for code rationale information?}
  \end{itemize}

We describe our methodology designed to address these research questions in \autoref{sec:Methodology} and answer the research questions using the results of the evaluation in \autoref{sec:Results}.
We discuss the threats to validity of our evaluation in \autoref{sec:Threads-to-Validity}.

%%%%%%%%%%%%%%%%%%%%%%%%%%%%%%%%%%%%%%%%%%%%%%%%%%%%%%%%%%%%%%%%%%%%%%
\section{Method}
\label{sec:Methodology}

We designed the user study as an explorative, semi-structured interview composed of two parts with an estimated duration of 1 to 1.5 hours.
A semi-structured interview format allows us to maintain a focused discussion while also eliciting interesting responses and reasoning from participants \cite{shull_guide_2007}.
The semi-structured interview format is also appropriate as we seek to understand how participants interact with the commit history for a file and the Intelligent History plugin.
The study involved having participants explore the commit history of two Java classes from the Apache Kafka project to find answers to questions related to motivation and rationale behind changes in the classes.
The briefing for the study and the setup instructions can be found in Appendix \ref{sec:Briefing-and-Setup}.
We expected the participants to interact with commit and issue information found in the commit history in order to be able to answer these questions.

The first part of the user study session consisted of asking questions from two sets of open-ended, task-oriented questions, which are presented in \autoref{tab:Question-Sets}.
We prepared two sets of questions, each pertaining to a different Java class from the Apache Kafka project.
This would facilitate an experience where the participant would complete an initial set of questions without being allowed to use the features of Intelligent History plugin.
After completing the initial set of questions, the participant would be introduced to the features of the Intelligent History plugin and permitted to use them to help them answer the second set of questions.
As shown in \autoref{tab:Question-Sets}, set A asks questions regarding the motivation behind particular changes in \class{Toplogy}, while set B asks the participant to identify potentially important or meaningful commits from the history and about motivation behind a specific method in \class{StreamsBuilder}.
The \class{Topology} class contains 22 commits and the \class{StreamsBuilder} class contains 45 commits respectively.
We limited the number of questions per class based on the number of commits in each class' commit history and to accomodate participants' varying preferences for and familiarity with exploring commit history in general.

We shuffle the set of questions such that half the participants received and completed set A first \emph{without} awareness of the plugin, and set B afterwards \emph{with} introduction to the plugin.
The remaining half of the participants received set B first \emph{without} the plugin, and set A \emph{with} the plugin.
We perform this shuffling of question sets to avoid having the features of Intelligent History be more favourable for one set of questions to another.
There is some overlap between the sets, particularly with questions A(b) and B(b) from \autoref{tab:Question-Sets}.
However, we anticipate the features of Intelligent History to be most useful for participants that receive set B as the question set they will be allowed to use the Intelligent History plugin to aid their search for answering.
In particular, question B(a) from \autoref{tab:Question-Sets} expects the participant to examine the diffs, messages, and Jira issue for each commit in order to identify two changes that they can justify have a certain intent, which is to improve some functionality in \class{StreamsBuilder}. 
Other questions such as 
This will allow us to further understand how Intelligent History performs for different tasks and how Intelligent History fits to the participants' individual approaches to realistic tasks.

\begin{table}[]
  \centering
  \begin{tabular}{@{}ccl@{}}
  \toprule
  Set                                      & Class Name                                                   & \multicolumn{1}{c}{Question}                                                                                                                                                                                                                                                                                                                                            \\ \midrule
  \multicolumn{1}{|c|}{\multirow{3}{*}{A}} & \multicolumn{1}{c|}{\multirow{3}{*}{\class{Topology}}}       & \multicolumn{1}{p{8cm}|}{\begin{tabular}[c]{@{}p{8cm}}\small (a) Can you describe the motivation behind why the code segment on lines 717 to 722 was introduced and the benefit to the user of the Kafka \entity{API}?\end{tabular}}                                                                                                                                                             \\ \cmidrule(l){3-3} 
  \multicolumn{1}{|c|}{}                   & \multicolumn{1}{c|}{}                                        & \multicolumn{1}{p{8cm}|}{\begin{tabular}[c]{@{}p{8cm}}\small (b) There are several overloaded methods called \code{addSink} in this class. Can you describe in what context were these overloaded methods introduced to this class?\end{tabular}}                                                                                                                                             \\ \cmidrule(l){3-3} 
  \multicolumn{1}{|c|}{}                   & \multicolumn{1}{c|}{}                                        & \multicolumn{1}{p{8cm}|}{\small (c) What was the motivation behind the most recent commit (\commit{fbe7fb94}) made to this class?}                                                                                                                                                                                                                                                               \\ \midrule
  \multicolumn{1}{|c|}{\multirow{2}{*}{B}} & \multicolumn{1}{c|}{\multirow{2}{*}{\class{StreamsBuilder}}} & \multicolumn{1}{p{8cm}|}{\begin{tabular}[c]{@{}p{8cm}}\small (a) Can you find two commits that introduced changes to improve some functionality in this class and justify why you chose them? The changes can not be cosmetic such as removing repeated words, removing deprecated code, or adding/modifying/removing documentation and comments.\end{tabular}} \\ \cmidrule(l){3-3} 
  \multicolumn{1}{|c|}{}                   & \multicolumn{1}{c|}{}                                        & \multicolumn{1}{p{8cm}|}{\small (b) Why is the \code{build} method in this class overloaded? Specifically, why was the overloaded build method in lines 623 to 630 introduced?}                                                                                                                                                                                                                \\ \bottomrule
  \end{tabular}
  \caption{
    The question sets used in the user study. 
    Set A pertains to the \class{Topology} class and set B relates to the \class{StreamsBuilder} class.
    }
  \label{tab:Question-Sets}
\end{table}

The second part of the session consisted of asking a series of open-ended interview questions to gather information about the participant's background with respect to software development and experience in examining commits and issues from an issue tracking system (\entity{ITS}).
We also asked participants to compare their experience in completing the question sets without and with the use of Intelligent History.
The interview questions are available at Appendix \ref{sec:Interview-Questions}.
As part of the semi-structured nature of the study, the author of this thesis as the interviewer would also ask improvised follow-up questions related to the participants' experience with using revision history and obtaining code rationale information. 

The interview sessions took place virtually over Zoom.
We recorded all of the sessions for manual review and documented each session with a summary. 

\subsection{Participants}

We recruited 10 participants through public posting on social media and circulating a letter of initial contact within the author's professional network.
We focused on recruiting software developers with at least 1 year of experience in software development, which could include non-professional and professional experience.
Participants were compensated with entrance to a raffle for 1 of 5 Amazon gift cards valued at $\$40$ \entity{CAD}.
\autoref{tab:Participants} shows the demographic information we collected for each participant and the order in which each participant received the question sets from \autoref{tab:Question-Sets}.
The role for each participant indicates their occupation title at the time of the user study.
We generalized the participants' roles to ensure their anonymity.
Among 10 participants, 4 were students and 6 were full-time developers.
In terms of the participants' total years of experience in software development, the mean is 5.2 years and the median is 4.3 years.

\begin{table}[h]
  \centering
  \begin{tabular}{@{}cclcc@{}}
  \toprule
  \multicolumn{1}{l}{Pseudo-initial} & \multicolumn{1}{l}{YoE (p/n-p)} & \multicolumn{1}{c}{Role} & \multicolumn{1}{l}{First Set} & \multicolumn{1}{l}{Second Set} \\ \midrule
  A                                  & 4.0 (1.0/3.0)                     & Firmware Developer       & A                                  & B                               \\
  B                                  & 3.0 (2.8/0.2)                     & Software Developer       & B                                  & A                               \\
  C                                  & 6.0 (3.0/3.0)                     & UX Engineer              & A                                  & B                               \\
  D                                  & 4.5 (2.5/2.0)                     & Software Developer       & B                                  & A                               \\
  E                                  & 13.0 (6.0/7.0)                    & Graduate Student         & A                                  & B                               \\
  F                                  & 6.0 (2.0/4.0)                     & Software Developer       & B                                  & A                               \\
  G                                  & 2.0 (0.0/2.0)                     & Undergraduate Student    & A                                  & B                               \\
  H                                  & 3.5 (1.5/2.0)                     & Software Developer       & B                                  & A                               \\
  I                                  & 4.0 (3.0/1.0)                     & Graduate Student         & A                                  & B                               \\
  J                                  & 6.0 (2.0/4.0)                     & Graduate Student         & B                                  & A                               \\ \bottomrule
  \end{tabular}
  \caption{
    Participants by pseudo-initial, total years of experience (YoE) with separation by professional (p) and non-professional (n-p) experience, current role, and the question set the participant received first and second.
    The participant answers the first set of questions \emph{without} using the features of Intelligent History and the second set of questions \emph{with} the support of the plugin.
  }
  \label{tab:Participants}
\end{table}

\subsection{Analysis}

We instrumented Intelligent History to log timestamps and events occuring within the IDE.
In particular, we logged the commits a participant examined for each question and the features of Intelligent History that the participant invoked.
This would allow us to obtain quantitative information on the number of commits a participant investigated to answer a question and how the participant interacted with the Intelligent History plugin for answering the question set in which they were permitted to use its features.
On the basis of the session recordings and logs, we constructed graphs for each participant's session to visualize how they explored commit history to answer the question sets and the impact of Intelligent History on their exploration.
\FIXME{Include sample commit/exploration graphs.}

From conducting a follow-up interview of each participant after they completed the question sets, we obtained qualitative information regarding each participant's background and familiarity with working in a project that uses an issue tracking system (\entity{ITS}) and their thoughts and experience with using Intelligent History.
\FIXME{Explain how the commit graphs were used to formulate themes related to commit history exploration and also created themes based on the qualitative responses from participants.}

%%%%%%%%%%%%%%%%%%%%%%%%%%%%%%%%%%%%%%%%%%%%%%%%%%%%%%%%%%%%%%%%%%%%%%
\section{Results}
\label{sec:Results}

\subsection{Summary}

\FIXME{Comment on the results and how the figures were produced to convey them.}

\begin{landscape}
\begin{table}[p]
  \begin{tabular}{@{}cccccccccccccccc@{}}
    \toprule
    \multicolumn{1}{l}{}       & \multicolumn{9}{c}{Set A (without plugin)}                                                                                                                                              & \multicolumn{6}{c}{Set B (with plugin)}                                                                                                     \\ \midrule
                               & \multicolumn{3}{c}{a} & \multicolumn{3}{c}{b}                                                          & \multicolumn{3}{c}{c}                                                          & \multicolumn{3}{c}{a}                                      & \multicolumn{3}{c}{b}                                                          \\ \midrule
    Pseudo-initial             & \%CE  & \%HC  & \#AS  & \multicolumn{1}{l}{\%CE} & \multicolumn{1}{l}{\%HC} & \multicolumn{1}{l}{\#AS} & \multicolumn{1}{l}{\%CE} & \multicolumn{1}{l}{\%HC} & \multicolumn{1}{l}{\#AS} & \%CE & \multicolumn{1}{l}{\%HC} & \multicolumn{1}{l}{\#AS} & \multicolumn{1}{l}{\%CE} & \multicolumn{1}{l}{\%HC} & \multicolumn{1}{l}{\#AS} \\ \midrule
    A                          & 40.9  & 55.6  & 1     & 27.3                     & 83.3                     & 1                        & ?                        & ?                        & ?                        & 15.6 & 100.0                    & 0                        & 6.7                      & 100.0                    & 0                        \\
    C                          & 4.5   & 83.3  & 1     & 27.3                     & 83.3                     & 1                        & ?                        & ?                        & ?                        & 6.7  & 100.0                    & 0                        & 0.0                      & 0.0                      & 0                        \\
    E                          & 36.3  & 62.5  & 1     & 68.1                     & 66.7                     & 1                        & ?                        & ?                        & ?                        & 35.6 & 100.0                    & 1                        & 8.9                      & 100.0                    & 0                        \\
    G                          & 86.3  & 57.9  & 1     & 36.3                     & 75.0                     & 1                        & ?                        & ?                        & ?                        & 13.3 & 100.0                    & 3                        & 6.7                      & 100.0                    & 1                        \\
    I                          & 4.5   & 83.3  & 1     & 36.3                     & 75.0                     & 1                        & ?                        & ?                        & ?                        & 6.7  & 100.0                    & 0                        & 6.7                      & 100.0                    & 0                        \\ \midrule
    \multicolumn{1}{l}{Mean}   & 34.5  & 68.5  & 1     & 39.1                     & 76.7                     & 1                        & ?                        & ?                        & ?                        & 15.6 & 100.0                    & 0.8                      & 5.8                      & 80.0                     & 0.2                      \\
    \multicolumn{1}{l}{Median} & 36.3  & 62.5  & 1     & 36.3                     & 75.0                     & 1                        & ?                        & ?                        & ?                        & 13.3 & 100.0                    & 0.0                      & 6.7                      & 100.0                    & 0.0                      \\ \bottomrule
  \end{tabular}
  \caption{
    Summary of results for participants who received question set A first to complete without the plugin and question set B after to complete with the plugin.
    For each question set and question, the table shows the percentage of commits from the commit history that the participant examined (\%CE), 
    the percentage of the participant's examined commits that would be or were highlighted by the plugin (\%HC),
    and the number of application switches between the \entity{IDE} and the browser (\#AC) that the participant made.
  }
  \label{tab:Results-Quantitative-AB}
\end{table}
\end{landscape}

\begin{landscape}
  \begin{table}[p]
    \begin{tabular}{@{}cccccccccccccccc@{}}
      \toprule
      \multicolumn{1}{l}{}       & \multicolumn{9}{c}{Set A (with plugin)}                                                                                                                                                 & \multicolumn{6}{c}{Set B (without plugin)}                                                                                                  \\ \midrule
                                 & \multicolumn{3}{c}{a} & \multicolumn{3}{c}{b}                                                          & \multicolumn{3}{c}{c}                                                          & \multicolumn{3}{c}{a}                                      & \multicolumn{3}{c}{b}                                                          \\ \midrule
      Pseudo-initial             & \%CE  & \%HC  & \#AS  & \multicolumn{1}{l}{\%CE} & \multicolumn{1}{l}{\%HC} & \multicolumn{1}{l}{\#AS} & \multicolumn{1}{l}{\%CE} & \multicolumn{1}{l}{\%HC} & \multicolumn{1}{l}{\#AS} & \%CE & \multicolumn{1}{l}{\%HC} & \multicolumn{1}{l}{\#AS} & \multicolumn{1}{l}{\%CE} & \multicolumn{1}{l}{\%HC} & \multicolumn{1}{l}{\#AS} \\ \midrule
      B                          & 4.5   & 100.0 & 1     & 9.0                      & 100.0                    & 0                        & ?                        & ?                        & ?                        & 35.6 & 37.5                     & 2                        & 4.4                      & 100.0                    & 0                        \\
      D                          & 31.8  & 100.0 & 1     & 13.3                     & 100.0                    & 1                        & ?                        & ?                        & ?                        &      &                          &                          &                          &                          &                          \\
      F                          &       &       &       &                          &                          &                          & ?                        & ?                        & ?                        &      &                          &                          &                          &                          &                          \\
      H                          &       &       &       &                          &                          &                          & ?                        & ?                        & ?                        &      &                          &                          &                          &                          &                          \\
      J                          &       &       &       &                          &                          &                          & ?                        & ?                        & ?                        &      &                          &                          &                          &                          &                          \\ \midrule
      \multicolumn{1}{l}{Mean}   &       &       &       &                          &                          &                          & ?                        & ?                        & ?                        &      &                          &                          &                          &                          &                          \\
      \multicolumn{1}{l}{Median} &       &       &       &                          &                          &                          & ?                        & ?                        & ?                        &      &                          &                          &                          &                          &                          \\ \bottomrule
    \end{tabular}
    \caption{
      Summary of results for participants who received question set B first to complete without the plugin and question set A after to complete with permitted use of the plugin.
      For each question set and question, the table shows the percentage of commits from the commit history that the participant examined (\%CE), 
      the percentage of the participant's examined commits that would be or were highlighted by the plugin (\%HC),
      and the number of application switches between the \entity{IDE} and the browser (\#AC) that the participant made.
    }
    \label{tab:Results-Quantitative-BA}
  \end{table}
  \end{landscape}

\subsection{RQ1}

\subsection{RQ2}

\subsection{RQ3}

%%%%%%%%%%%%%%%%%%%%%%%%%%%%%%%%%%%%%%%%%%%%%%%%%%%%%%%%%%%%%%%%%%%%%%
\section{Threats to Validity}
\label{sec:Threads-to-Validity}

\FIXME{Hawthorne effect, sample size, tool usage for second set of questions -- we did this to mitigate bias or impression of conducting the study a certain way, lack of external validity due to the controlled environment with prescribed questions.}

%%%%%%%%%%%%%%%%%%%%%%%%%%%%%%%%%%%%%%%%%%%%%%%%%%%%%%%%%%%%%%%%%%%%%%
\endinput

Any text after an \endinput is ignored.
You could put scraps here or things in progress.