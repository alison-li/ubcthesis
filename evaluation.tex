\chapter{Evaluation}
\label{ch:Evaluation}

We hypothesized identifying important commits automatically and providing integrated access between commit and issue information could help a developer more effectively access the rationale for code.
To evaluate our hypothesis, we conducted a user study in the form of a semi-structured interview and devised the following research questions:

\begin{itemize}[leftmargin=*]
    \item[] \label{itm:RQ1} \textbf{Research Question 1 (\entity{RQ1}):} \textit{Does highlighting important commits help software developers explore a file’s revision history more efficiently?}
    \item[] \label{itm:RQ2} \textbf{Research Question 2 (\entity{RQ2}):} \textit{Does highlighting important commits cause software developers to be more efficient at exploring a file’s revision history than a conventional linear, temporal, flat view of revision history?}
    \item[] \label{itm:RQ3} \textbf{Research Question 3 (\entity{RQ3}):} \textit{How useful is direct integration of issue information in the IDE for helping a developer search for code rationale information?}
  \end{itemize}

We describe our methodology designed to address these research questions in \autoref{sec:Methodology} and answer the research questions using the results of the evaluation in \autoref{sec:Results}.
We discuss the threats to validity of our evaluation in \autoref{sec:Threads-to-Validity}.

%%%%%%%%%%%%%%%%%%%%%%%%%%%%%%%%%%%%%%%%%%%%%%%%%%%%%%%%%%%%%%%%%%%%%%
\section{Methodology}
\label{sec:Methodology}

We conducted semi-structured interviews, which involved asking a series of open-ended, task-oriented questions from two questions sets to maintain a concentrated discussion while also permitting
interesting responses and reasoning behind the responses \cite{shull_guide_2007}.

\subsection{Question Sets}

Task sets and evaluation criteria.

\subsection{Measures}

\subsection{Participants}

We recruited $N$ participants, focusing on software developers with at least 1 year of experience in software development.
Participants were compensated with entry to a raffle to win 1 of 5 Amazon gift cards valued at $\$40$ \entity{CAD} each for a 1 to 1.5 hour session.
\FIXME{Describe years of experience.}

\subsection{Procedure}

%%%%%%%%%%%%%%%%%%%%%%%%%%%%%%%%%%%%%%%%%%%%%%%%%%%%%%%%%%%%%%%%%%%%%%
\section{Results}
\label{sec:Results}

\subsection{Quantitative Analysis}

\subsection{Qualitative Analysis}

%%%%%%%%%%%%%%%%%%%%%%%%%%%%%%%%%%%%%%%%%%%%%%%%%%%%%%%%%%%%%%%%%%%%%%
\section{Threats to Validity}
\label{sec:Threads-to-Validity}

%%%%%%%%%%%%%%%%%%%%%%%%%%%%%%%%%%%%%%%%%%%%%%%%%%%%%%%%%%%%%%%%%%%%%%
\endinput

Any text after an \endinput is ignored.
You could put scraps here or things in progress.