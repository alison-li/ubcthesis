\chapter{Evaluation}
\label{ch:Evaluation}

We hypothesized identifying important commits automatically and providing integrated access between commit and issue information could help a developer more effectively access the rationale for code.
To evaluate our hypothesis, we conducted a user study in the form of a semi-structured interview and devised the following research questions:

\textbf{Research Question 1 (RQ1):} \textit{Does highlighting important commits help software developers explore a file’s revision history more efficiently?}

\textbf{Research Question 2 (RQ2):} \textit{Does highlighting important commits cause software developers to be more efficient at exploring a file’s revision history than a conventional linear, temporal, flat view of revision history?}

\textbf{Research Question 3 (RQ3):} \textit{How useful is direct integration of issue information in the IDE for helping a developer search for code rationale information?}

%%%%%%%%%%%%%%%%%%%%%%%%%%%%%%%%%%%%%%%%%%%%%%%%%%%%%%%%%%%%%%%%%%%%%%
\section{Methodology}
\label{sec:Methodology}

We conducted semi-structured interviews, involving asking a series of open-ended, task-oriented questions to maintain a concentrated discussion while also permitting
interesting responses and reasoning behind the responses \cite{shull_guide_2007}.

%%%%%%%%%%%%%%%%%%%%%%%%%%%%%%%%%%%%%%%%%%%%%%%%%%%%%%%%%%%%%%%%%%%%%%
\section{Results}
\label{sec:Results}

%%%%%%%%%%%%%%%%%%%%%%%%%%%%%%%%%%%%%%%%%%%%%%%%%%%%%%%%%%%%%%%%%%%%%%
\section{Threats to Validity}
\label{sec:Threads-to-Validity}

%%%%%%%%%%%%%%%%%%%%%%%%%%%%%%%%%%%%%%%%%%%%%%%%%%%%%%%%%%%%%%%%%%%%%%
\endinput

Any text after an \endinput is ignored.
You could put scraps here or things in progress.