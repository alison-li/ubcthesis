\chapter{Conclusion}
\label{ch:Conclusion}

The navigation of large software revision histories, compounded by a software
project's association with an equally large issue repository or issue tracking system
presents a cognitive and temporal burden for software developers seeking 
answers to code rationale questions.
While issues are a significant source of code rationale information that
can help to contextualize code changes with motivational information,
developers must overcome the challenges of manually determining
which commits in a large history are relevant and worth further investigation.
Moreover, since issues are distributed across an issue tracking system,
developers must context switch between the application they use to view
source code revision history \rev{and the} application they use to access a project's issue tracking system.

Our goal was to reduce the cognitive and temporal costs of software history exploration.
To that end, we presented Intelligent History, a plugin for the IntelliJ IDEA \entity{IDE}
as an implementation of an approach to (1) automatically distinguish potentially interesting
commits from less important commits and (2) provide better access to issue information 
when examining commits in a revision history.
Using a heuristics-based approach with regular expression patterns, 
we implemented Intelligent History to use commit highlighting 
to make the distinction between commits that it identifies as containing
only non-essential changes and other commits, which might be potentially more meaningful to examine.
We also automatically extract referenced Jira issue \entity{ID}s from a commit message title
and display the issue title and description, along with metadata information, directly in the \entity{IDE}.

We found \rev{that} the results of our evaluation of Intelligent History show
potential for our approach as there was overlap between at least half of the commits
that participants examined without Intelligent History and the commits that Intelligent History
highlighted. Participants who used Intelligent History examined significantly
\rev{fewer} commits than participants who did not use Intelligent History.
Lastly, we found that direct integration of issue information in the \entity{IDE}
with Intelligent History might support developers who employ a commit history exploration approach
we describe as \textit{linear} or \textit{cyclic and backtracking} to contextualize
source code changes quickly and view slightly more issues than if they did not have
direct issue integration.
Our analysis demonstrates promise for supporting developers in more
efficiently exploring a commit history to answer code rationale questions.