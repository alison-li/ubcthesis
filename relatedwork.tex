\chapter{Related Work}
\label{ch:Related-Work}

Rationale questions ask why code was implemented a certain way, 
why alternative implementations were not chosen, and what the hidden criteria might have been 
behind motivating design choices about code \cite{latoza_hard-answer_2010}.
Many empirical studies have established code rationale as a highly sought after category of 
information that lacks effective tool support to answer them \cite{latoza_maintaining_2006, latoza_hard-answer_2010, ko_information_2007}. 
From a survey of 157 software developers, LaToza \etal found 66\% of respondents report understanding 
the intent behind a piece of code to be a serious problem, 
while 82\% report that it takes considerable effort to understand ``why the code is implemented the way it is'' \cite{latoza_maintaining_2006}.
Further, 51\% agree that understanding the history of a piece of a code to be a serious problem.
In support of the findings by LaToza et al., the observational study by Ko \etal of 17 software 
developers found it was rarely enough for developers to understand the cause of a program's behaviour, 
as developers also sought after the historical reason for a program's current implementation \cite{ko_information_2007}.

Two particular sources of information that are helpful for answering code rationale questions 
are a software project's revision history and the issues from a project's issue tracking system (\entity{ITS}). 
In the absence of knowledgeable colleagues, several studies observe and show consensus 
on developers seeking answers to code rationale questions through code change histories 
and bug reports \cite{ko_information_2007, codoban_software_2015, robillard_turnover-induced_2021}.

%%%%%%%%%%%%%%%%%%%%%%%%%%%%%%%%%%%%%%%%%%%%%%%%%%%%%%%%%%%%%%%%%%%%%%

\pagebreak
\section{Existing Support for Revision History Exploration}

Version control systems (\entity{VCS}) allow developers to retain historical versions of source
code and project files under development for retrieval and review \cite{ruparelia_history-vcs_2010}.
History exploration tools exist to make exploring the history captured in \entity{VCS} more
convenient for developers.
However, through interviews and surveys, Codoban \etal discovered existing history exploration tools 
to be limited by only presenting a linear, sequential view of history \cite{codoban_software_2015}.
Current tools related to \entity{vcs} assume all history is equally important 
with a flat, non-hierarchical structure and only provide a temporal view of committed changes \cite{codoban_software_2015}.
In other words, these tools only offer a time-based view of software revision history.
Since a time-based view alone considers all commits to be equally important,
developers face the challenge of having to navigate revision histories containing 
large quantities of commits, \rev{many of which are not relevant to the information they seek}.

For developers who use revision history as a source of information
for recovering rationale behind code and for understanding the evolution of code, 
developers require more than a time-based view alone to effectively 
and effficiently support their needs \cite{codoban_software_2015}.
A developer seeking to understand the evolution of a particular file must 
manually identify meaningful, non-trivial commits and mentally keep track of related
changes across commits in a history.
Some commits may or may not contain any rationale information in
commit messages or referenced artifacts, like issues and design documents.
Codoban \etal suggest that providing better grouping capabilities for
grouping related commits and better filtering capabilities for commit types (\eg only showing non-merge commits) and diffs
can help developers mitigate these challenges to exploring software history.

Existing \entity{git} commands as tools for exploring revision history such 
as ``blame'' \cite{gitblame} annotate each line in a file with information about the revision 
and the author that last modified a particular line.
If a developer was investigating the design choice for a particular variable's value, 
the most recent revisions that last affected a particular line might not give this insight.
For example, the most recent changes could be fixing a typo in the variable's name or the most recent 
few revisions affecting the line could be due to moving the declaration of the variable as part 
of a refactoring or modification in other parts of the file.
In this case, the developer would need to go through several iterations of running ``blame'' 
to find the commit that first introduced the variable and hope the commit contains information 
about the choice behind the value set for it.
If the variable's value was changed several times across different revisions, 
then to further investigate what was changed, 
a developer would also have to use \entity{git} ``diff'' \cite{gitdiff} to show the 
changes between two revisions or two versions of a file.
However, the participants in the study by Codoban \etal reported diffs were difficult 
to read due to the challenge of gaining any insight or intent of the change from a diff 
and needing to go through numerous diffs to understand a change \cite{codoban_software_2015}.

Graphical user interfaces (\entity{gui}) for \entity{vcs} like GitExtensions \cite{gitextensions} 
and Sourcetree \cite{sourcetree} streamline a view of revisions alongside diffs, 
saving the user from manually executing calls of \entity{git} ``blame'' and ``diff.'' 
Native \entity{git} integrations and third-party plugins for popular code editors and IDEs 
have made the use of these \entity{git} features more convenient. 
For example, IntelliJ IDEA \cite{intellij} has a tool window for displaying the \entity{git} 
history of a project, similar to calling \entity{git} ``log.''
IntelliJ also has a tool window tab specifically to show the commit history 
for a single file or directory \cite{intellij-showhistory}.
Third-party plugins such as GitToolBox \cite{gittoolbox} for IntelliJ and GitLens \cite{gitlens} 
for Visual Studio Code also integrate information from \entity{git} ``blame'' by directly 
inlining a preview of the commit message and the author of the revision that last modified the line in a file.

While \entity{gui}s make it more convenient to inspect revisions one at a time and refer 
or jump to different revisions at different points in time, they do not address the concern 
of having to go through several diffs
in order to retrieve knowledge buried in software history. 
Many diffs can be irrelevant to a code segment of interest, 
making it time-consuming to find a meaningful explanation for the change.
Moreover, plugins that make the information retrieved from \entity{git} commands more 
accessible still only cover the temporal view of software history.
A developer would nonetheless need to go through several commits and diffs and meanwhile, 
several of the encountered commits could be trivial changes regarding the section of code 
or even the file that they are interested in.
As 58\% of survey respondents from the study by Codoban \etal report
that understanding the evolution of code to be a primary reason for
referring to old history and expressed the desire to
understand how previous versions of code relate to present code \cite{codoban_software_2015},
the provision of more strategic ways for viewing, filtering, and 
navigating revision history beyond a temporal view would benefit developers by 
making the exploration of revision history more efficient.

%%%%%%%%%%%%%%%%%%%%%%%%%%%%%%%%%%%%%%%%%%%%%%%%%%%%%%%%%%%%%%%%%%%%%%

\section{Improving Revision History and Exploration}

There have been many approaches to improving a developer's experience when exploring revision history in a software project. 
While some approaches in existing work do not focus on supporting code rationale questions, 
they all address the concern of improving the revision history for code changes in a software project.

Kawrykow and Robillard propose DiffCat \cite{kawrykow_non-essential_2011}, 
a tool for detecting non-essential changes in software history and therefore an approach to finding 
more meaningful representations of software changes.
In their work, ``non-essential'' changes are characterized as changes which are cosmetic in nature, 
generally behaviour-preserving, and unlikely to yield further insight into the roles of or 
relationships between the program entities they modify.
\rev{
As part of their implementation for detecting non-essential changes, 
Kawrykow and Robillard catalog trivial type declaration updates made to entities,
local variable extractions (\ie the use of temporary variables to store expressions),
rename-induced modifications, trivial keyword modifications (\eg \code{this} in Java), 
local variable renames, and whitespace and documentation-related updates.
Kawrykow and Robillard evaluate DiffCat by measuring the prevalence and
impact of non-essential differences in revision histories on code churn and method updates.
As part of the implementation for detecting non-essential changes, Kawrykow and Robillard use
a Partial Program Analysis (\entity{PPA}) technique that works with Abstract Syntax Tree (\entity{AST}) representations of
Java and \entity{AST}-differencing.
}

The approach in this thesis explores the use of \rev{a subset of their catalog of trivial changes
as heuristics} for \rev{the purpose of} reducing 
the number of trivial commits a developer must go through in a file's revision history when searching 
for code rationale information on why the code in a file looks the way it does.
\rev{Our subset of non-essential changes includes changes related to documentation, whitespace,
newlines, import statements and specific Java keywords such as annotations.}
Our work is focused on investigating the efficacy of eliminating non-essential changes
to support software history exploration for obtaining code rationale information
\rev{and uses an approach incorportating regular expression patterns for detecting
trivial changes as a lightweight technique to support efficient use for developers.}

Bradley and Murphy \cite{bradley_supporting_2011} investigate two user interfaces, Deep Intellisense and Rationalizer, 
for supporting software history exploration.
They introduce a model for expressing how software repository information is interconnected 
and use the model for comparing the effectiveness of two different presentation styles.
In the Rationalizer interface, historical information is directly integrated into the background of the source code editor.
This historical information addresses the \emph{when}, \emph{who}, and \emph{why} for each line of code in a file 
and provides hyperlinks to relevant Bugzilla reports \cite{bradley_supporting_2011}.
The \emph{why} column in Rationalizer extracts the message from the last revision that modified a line, 
which may not adequately cover the intent behind a code change as commit messages 
might only describe the \emph{what} of a change at a high level.
In the worst case, commit messages may not contain any information.
The analysis by Dyer \etal of over 23,000 Java SourceForge projects found 14\% of all 
commit messages to be completely empty \cite{dyer_boa_2013}.

For reducing the volume of commits a developer must explore, Chronos \cite{servant_history_2012} uses history slicing, 
a technique for identifying the minimal number of code modifications across any number of revisions for any segment of code, 
and offers a visualization of the evolution of any piece of code in a software project's code base.
For reasoning about the evolution of source code, CodeShovel \cite{grund_codeshovel_2021} 
uses revision history to construct method-level histories where the revisions affecting a 
method have been labelled with change categories.

%%%%%%%%%%%%%%%%%%%%%%%%%%%%%%%%%%%%%%%%%%%%%%%%%%%%%%%%%%%%%%%%%%%%%%

\section{Incorporating Issue Tracking Systems}

Previous work has explored using non-source code project artifacts to enrich existing code 
rationale information embedded in source code. 
A common source of code rationale information is a project's issue repository because these 
issues can contain useful discussion in the form of comments such as the hypothesized causes of bugs, 
approaches to fixing the bugs, and the motivation behind a feature implementation. 
Issue repositories like Jira also support extensive organization for issues,
which can designate Jira issues belonging to different levels of abstraction.
For example, an issue on Jira can be categorized as a bug, task or sub-task of a larger task, story, or epic \cite{jira-issue-types}.
Jira issues can also be linked to each other by relationships such as ``caused by'' or ``relates to.''

Rastkar and Murphy \cite{rastkar_why_2013} applied multi-document summarization techniques on Jira issues 
at different levels of abstraction to generate summaries providing motivational information behind a code change.
While Rastkar and Murphy's approach is a step towards finding and providing developers with the ``why'' of code changes, 
the approach alone does not provide the ability to view and explore commits alongside the summaries,
resulting in a cognitive burden for the developer to mentally 
keep track of and relate source code changes to rationale information.

Another tool recognizing Jira as a source of code rationale information is \entity{ARENA} \cite{moreno_arena_2017}, 
which aims to answer the \emph{what} and \emph{why} of code change between releases by 
using the commits in a project's versioning system and the issues from the project's Jira instance 
to automatically generate release notes. 
\entity{ARENA} summarizes related sets of code changes between releases to answer \emph{what} 
changed in a newer release and uses change categorizations (\eg improvement, bug fix, new feature, \etc) 
and Jira issue titles to account for the \emph{why} of code change between releases.
However, a shortcoming raised in the evaluation of \entity{ARENA} is that the approach 
relies on automatically linking issues to commits and focuses on 
using the titles of issues to document the rationale behind changes.
While \entity{ARENA} is focused on the automatic generation of release notes for structural changes,
we are primarily concerned with real-time co-exploration of source code revision history 
and code rationale information.

%%%%%%%%%%%%%%%%%%%%%%%%%%%%%%%%%%%%%%%%%%%%%%%%%%%%%%%%%%%%%%%%%%%%%%

\section{Summary}

One facet of previous work has focused on improving software history exploration by making revision history 
a primary source of code change information easier to navigate and obtain the \emph{what} of code change.
While this work has made it more convenient to explore software history by reducing the search space of changes, 
the cognitive burden of understanding the change still lies with the developer.
Using only the source changes contained in revisions and the commit messages in a revision history 
means any generated outcome lacks the context a change took place in.
Meanwhile, the other side of previous work incorporates a project's issue repository for elaborating 
on the \emph{why} of code change but still lacks support for providing the developer with the efficiency to 
navigate commits themselves and associate meaningful changes with rationale.

This thesis explores the development of an approach for making the exploration of the revision history 
for a given file more efficient by emphasizing potentially more meaningful commits, while also making 
access to the Jira issue information relevant to each commit in a file's revision history more convenient.
Participants in the study by Codoban \etal noted searching through
each version of a file to indentify when particular features were
implemented \cite{codoban_software_2015}. 
Participants in their study also reported on a desire for 
better visualization of the history of a file.
Thus, we propose to operate at the file-level to scope the determination of
a commit as non-essential in terms of how the commit affects a file and
its evolution.

%%%%%%%%%%%%%%%%%%%%%%%%%%%%%%%%%%%%%%%%%%%%%%%%%%%%%%%%%%%%%%%%%%%%%%
\endinput

Any text after an \endinput is ignored.
You could put scraps here or things in progress.