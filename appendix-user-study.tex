\chapter{User Study Materials}

%%%%%%%%%%%%%%%%%%%%%%%%%%%%%%%%%%%%%%%%%%%%%%%%%%%%%%%%%%%%%%%%%%%%%%

\section{Briefing and Setup Instructions}
\label{sec:Briefing-and-Setup}

\subsection{Briefing}
\label{subsec:Briefing}

Apache Kafka is an event streaming platform used to collect, process, and store streaming event data that has no discrete beginning or end, \eg real-time stock trades and social media feeds. 
As part of Kafka’s contribution guidelines, a commit’s title/subject is usually of the form ``KAFKA-XXXX: Title,'' where ``KAFKA-XXXX'' is the relevant Jira issue ID and ``Title'' may be the Jira’s title or a more specific title describing the pull request itself. 
For trivial cases where a Jira issue is not required, ``MINOR:'' or ``HOTFIX:'' is used as the commit’s title prefix.

In this study, you will be exploring two Java classes and their commit histories from the Kafka project and answering a series of questions in a semi-structured interview format. The process of finding the answers to these questions can involve reading the commit messages for commits and visiting the relevant Jira issue on Apache’s Jira instance:
\url{https://issues.apache.org/jira/browse/KAFKA-XXXX}.

You may also see the acronym ``KIP'' referenced in some commits or Jira issues in Kafka. You may use and reference KIPs in your exploration and answers. These are Kafka Improvement Proposals (\entity{KIP}s) and can be found at: 
\url{https://cwiki.apache.org/confluence/display/kafka/kafka+improvement+proposals}.

\subsection{Setup Instructions}
\label{sec:Setup-Instructions}

\begin{enumerate}
    \item Install IntelliJ IDEA, either Community (free) or Ultimate edition (if you already have a license). See: \url{https://www.jetbrains.com/idea}. If you already have IntelliJ installed, please confirm the version number. The version must be 2022.1 or greater, which you can verify by navigating to the top bar  ``Help'' → ``About'' (Windows) or top bar → ``IntelliJ IDEA'' → ``About IntelliJ IDEA'' (Mac).
    \item Clone the Apache Kafka repository (roughly 200 MB on disk) or you can request the co-investigator to provide a ZIP of the repository: 
    
    \begin{center}
        \code{git clone --single-branch --branch trunk https://github.com/apache/kafka.git}
    \end{center}

    \item Open the Kafka project in IntelliJ: ``File'' → ``Open''
    \item To install the plugin, download \code{Intelligent.History-1.0.0.zip} at: \url{https://github.com/Alison-Li/intelligent-history/releases/tag/v1.0.0-eval}. Go to “Settings/Preferences” → ``Plugins'' → Gear Icon → ``Install Plugin from Disk\dots''
    \item For the task involving the plugin: Go to ``Settings/Preferences'' → ``Tools'' → ``Intelligent History'' to configure the Jira settings. Alternatively, go to “Settings/Preferences’ → type ``Intelligent History'' in the search field to look for the plugin settings. The co-investigator will provide you with the information needed to fill in the fields.
\end{enumerate}

%%%%%%%%%%%%%%%%%%%%%%%%%%%%%%%%%%%%%%%%%%%%%%%%%%%%%%%%%%%%%%%%%%%%%%

\section{Answer Key}
\label{sec:Answer-Key}

\begin{landscape}
    \begin{table}[]
        \centering
        \footnotesize
        \begin{tabular}{@{}cl@{}}
        \toprule
        Question & \multicolumn{1}{c}{Answer}                                                                                                                                                                                                                                                                                                                                                                                                                                                                                                                                                                                                                              \\ \midrule
        A1       & \begin{tabular}[c]{@{}l@{}}The participant should find commit \commit{075bbcfe} and reference the Jira issue KAFKA-7523.\\ The change implicitly connects state stores to processors/transformers to avoid burdening the API user\\ with having to explicitly specify the stores.\end{tabular}                                                                                                                                                                                                                                                                                                                                                                   \\ \midrule
        A2       & \begin{tabular}[c]{@{}l@{}}The participant should find commit \commit{f33e9a34} and reference the Jira issue KAFKA-4936.\\ The participant is expected to deliver a summary referencing the following motivation behind the change: \\ Allow users to dynamically choose which topic to send to at runtime, which can help remove the burden of having to update and restart KStreams applications. \end{tabular}                                                                                                                                                        \\ \midrule
        B1       & \begin{tabular}[c]{@{}l@{}}Possible answers:\\ \commit{075b368d} (KAFKA-6958) allows the user to define custom names for processors with the KStreams DSL to make complex topologies easier to understand.\\ \commit{3e64e5b9} (KAFKA-6761) reduces the footprint of Kafka Streams, which reduces resource utilization. \\ \commit{c0353d8d} (KAFKA-6036) optimizes when to use logical materialization, related to reducing Kafka Streams footprint. \\ The participant should choose commits which improve some functionality in the \class{StreamsBuilder} class\\ and justify their answer by understanding commit author intent.\\ If the commit intends to fix a bug, the participant should describe the bug and how the commit resolved the bug.\\ The participant may not choose commits which only modify comments, fix typos, add newlines.\end{tabular} \\ \midrule
        B2       & \begin{tabular}[c]{@{}l@{}}The participant should find commit \commit{e09d6d79} and reference the Jira issue KAFKA-7027.\\ The overloaded \code{build} method provides the option to accept user-defined configurations which have the required values for \\ building the Kafka topology.\\ This controls optimization by leaving the construction of the topology to the overloaded build \code{method}.\end{tabular}                                                                                                                                                                                                                                                        \\ \bottomrule
        \end{tabular}
    \end{table}
\end{landscape}

%%%%%%%%%%%%%%%%%%%%%%%%%%%%%%%%%%%%%%%%%%%%%%%%%%%%%%%%%%%%%%%%%%%%%%

\section{Interview Questions}
\label{sec:Interview-Questions}

\begin{enumerate}
    \item How many years of software development experience do you have, both professionally and non-professionally?
    \item In your development experience, have you often looked at commits and the issues associated with them (e.g. GitHub issues/pull requests, Jira issues, Bugzilla, etc.)?
        \begin{enumerate}
            \item In what scenarios or kinds of tasks do you typically look at commits and issues?
            \item How would you usually go about obtaining answers to questions about the intent behind the code you are reading?
        \end{enumerate}
    \item What types of code changes or content in diffs do you think are less useful or more useful when trying to understand why a developer wrote code a certain way or made a particular change? 
        \begin{enumerate}
            \item What changes in diffs do you think make a commit more important or less important to look at?
        \end{enumerate}
    \item For the task involving use of the plugin, which features did you find helpful/unhelpful to your exploration?
    \item Are there any comments or feedback you would like to provide?
\end{enumerate}

%%%%%%%%%%%%%%%%%%%%%%%%%%%%%%%%%%%%%%%%%%%%%%%%%%%%%%%%%%%%%%%%%%%%%%
\endinput